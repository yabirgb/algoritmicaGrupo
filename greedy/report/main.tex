\documentclass{article}
\usepackage[left=3cm,right=3cm,top=2.5cm,bottom=2cm]{geometry} % page settings
\usepackage{amsmath} % provides many mathematical environments & tools
\usepackage{amssymb}
\usepackage{amsfonts}
\usepackage[spanish]{babel}

\usepackage{multirow}

\usepackage{algorithm}
\usepackage{algpseudocode}
\usepackage{pifont}

\usepackage[utf8]{inputenc}
\setlength{\parindent}{0mm}

\usepackage[parfill]{parskip}

% Para el código
\usepackage{listings}
\usepackage{xcolor}
\definecolor{gray}{rgb}{0.5,0.5,0.5}
\newcommand{\n}[1]{{\color{gray}#1}}
\lstset{numbers=left,numberstyle=\small\color{gray}}

% Entorno para estilo de ejercicios
\setlength{\parindent}{0pt}

\usepackage{color}   %May be necessary if you want to color links
\usepackage{hyperref}
\hypersetup{
    colorlinks=true, %set true if you want colored links
    linktoc=all,     %set to all if you want both sections and subsections linked
    linkcolor=blue,  %choose some color if you want links to stand out
}

\usepackage{graphicx}
\usepackage{subfig}

\usepackage{listings,textcomp}
\title{Problema del viajante de comercio\\ Travelling salesman problem}
\author{José Antonio Álvarez Ocete - Norberto Fernández de la Higuera \\ Javier Gálvez Obispo - Yábir García Benchakhtir }
\date{\today}

% Custom
\providecommand{\abs}[1]{\lvert#1\rvert}
\setlength\parindent{0pt}
\definecolor{Light}{gray}{.90}
\newcommand\ddfrac[2]{\frac{\displaystyle #1}{\displaystyle #2}}

% Listings
\lstset{literate=   % listings config
  {á}{{\'a}}1 {é}{{\'e}}1 {í}{{\'i}}1 {ó}{{\'o}}1 {ú}{{\'u}}1
  {Á}{{\'A}}1 {É}{{\'E}}1 {Í}{{\'I}}1 {Ó}{{\'O}}1 {Ú}{{\'U}}1
  {à}{{\`a}}1 {è}{{\`e}}1 {ì}{{\`i}}1 {ò}{{\`o}}1 {ù}{{\`u}}1
  {À}{{\`A}}1 {È}{{\'E}}1 {Ì}{{\`I}}1 {Ò}{{\`O}}1 {Ù}{{\`U}}1
  {ä}{{\"a}}1 {ë}{{\"e}}1 {ï}{{\"i}}1 {ö}{{\"o}}1 {ü}{{\"u}}1
  {Ä}{{\"A}}1 {Ë}{{\"E}}1 {Ï}{{\"I}}1 {Ö}{{\"O}}1 {Ü}{{\"U}}1
  {â}{{\^a}}1 {ê}{{\^e}}1 {î}{{\^i}}1 {ô}{{\^o}}1 {û}{{\^u}}1
  {Â}{{\^A}}1 {Ê}{{\^E}}1 {Î}{{\^I}}1 {Ô}{{\^O}}1 {Û}{{\^U}}1
  {œ}{{\oe}}1 {Œ}{{\OE}}1 {æ}{{\ae}}1 {Æ}{{\AE}}1 {ß}{{\ss}}1
  {ű}{{\H{u}}}1 {Ű}{{\H{U}}}1 {ő}{{\H{o}}}1 {Ő}{{\H{O}}}1
  {ç}{{\c c}}1 {Ç}{{\c C}}1 {ø}{{\o}}1 {å}{{\r a}}1 {Å}{{\r A}}1
  {€}{{\EUR}}1 {£}{{\pounds}}1 {ñ}{{\~{n}}}1
}

\lstset{    %listings config
	language=C++,
	belowcaptionskip=1\baselineskip,
	breaklines=true,
	frame=L,
	xleftmargin=0.1in,
	%otherkeywords={},
	showstringspaces=false,
	backgroundcolor=\color{white},
	basicstyle=\footnotesize\ttfamily,
	keywordstyle=\bfseries\color{purple!90!black},
	commentstyle=\itshape\color{gray!85!},
	identifierstyle=\color{blue!80!black},
	stringstyle=\color{green!60!black},
}

\begin{document}

\maketitle
\newpage
\tableofcontents
\newpage

\section{Descripción del problema}

En el problema del viajante de comercio (\textit{Travelling Salesman
  Problem}) se nos dan las coordenadas de un conjunto finito de
ciudades y se nos pide encontrar un recorrido de las misma de forma
que la distancia que recorramos sea la menor posible.

\section{Soluciones greedy}

\subsection{Nearest Neighbor}

Teniendo como conjunto de ciudades $\Omega$ y la matriz $\mathcal{M}$ 
que contiene la distancia entre cada una de las ciudades, partimos de
$\mathcal{C}$  el conjunto de posibles candidatos (su valor inicial es
$\Omega$) y el conjunto solución $\mathcal{R}$ (se inicializa vacío).

Mediante ésta estructura el algoritmo se puede clasificar como greedy, 
teniendo en cuenta que a cada iteración del algoritmo se busca el óptimo 
local (la ciudad más cercana a la actual) y todos los candidatos usados 
pasan al conjunto solución $\mathcal{R}$, descartándolos del conjunto 
$\mathcal{C}$.


El funcionamiento del algoritmo sería el siguiente:

\begin{algorithm}[H]
\caption{Nearest Neighbor}
\begin{algorithmic}
\State $\mathcal{R}$ = [random()$\%|$ $\Omega$ $|$]
\State $\mathcal{C}$ - $\mathcal{R}$[0]
\For{pos in [0,n-2]}
\State bestPos = nearest($\mathcal{M}$[$\mathcal{R}$[pos]],$\mathcal{C}$)
\State $\mathcal{R}$.push($\mathcal{C}$[bestPos])
\State $\mathcal{C}$.erase(bestPos)
\EndFor
\State return $\mathcal{R}$
\end{algorithmic}
\end{algorithm}

Donde la función "nearest" devuelve la posición de la ciudad que, de entre 
todos los candidatos, está más cerca a la actual.

El algoritmo es O(n²) ya que tiene la misma estructura que el algoritmo de
ordenación por selección, donde se escogía el valor mínimo de las componentes 
que no se habían insertado todavía. En nuestro caso se escoge la ciudad que 
es la más cercana a la actual y se incluye en el conjunto solución, a partir 
del cual se repetirá el mismo procedimiento.

\subsection{Cheapest Insertion}

Si tomamos $\Omega$ el conjunto de ciudades, donde cada ciudad la
notamos como $\omega$ y $\mathcal{M}$ la matriz simetrica de la
distancia entre las ciudades.

Para este tomamos $<i,j>$ tales que $min(M) = M_{ij}$. Partimos de la
solución $\Gamma = \{ \omega_i, \omega_j \} $ y en cada iteración
buscamos la ciudad de índice $p$ tal que $\omega_p \notin \Gamma$ y que minimiza:

\[
  M_{ip} + M_{jp} - M_{ij}
\]

Una implementación en pseudo-codigo sería:

\begin{algorithm}[H]
\caption{Cheap Insert}
\begin{algorithmic}
\State i, j = nodes(min($M$))
\State path = [i, j]
\For{city in cities}
\If{city not in path}
\For{node in path}
\State minimizeDistance(node, city, node+1)
\EndFor
\State insert(path, T)
\EndIf
\EndFor
\State return path
\end{algorithmic}
\end{algorithm}

Este algoritmo es un algoritmo claramente greedy donde:

\begin{itemize}
\item Nuestro conjunto de candidatos es el conjunto $\Omega$.
\item Mantenemos una lista de candidatos ya usados.
\item Como criterio para determinar si $\Gamma$ es valido buscamos no
  tener ciclos y que no se repitan los nodos.
\item Nuestra función de selección es la definida en algoritmo
  anterior y que determina en cada situación cual es la menor
  distancia que podemos agregar.
\item La función que nos permite comparar soluciones es la distancia
  del camino final $\Gamma$
\end{itemize}

\subsection{Otro algoritmo posible}
\section{Comparativa de los algoritmos}
\end{document}